\documentclass{homework651}
%% Feel free to add your own commonly used commands to this file.

\newcommand{\Reals}{\ensuremath{\mathbb R}}% Gives you a shortcut for writing the blackboard R for the real numbers - \RR
\newcommand{\Nats}{\ensuremath{\mathbb N}} % Gives you a shortcut for writing the blackboard N for the natural numbers - \NN
\newcommand{\Ints}{\ensuremath{\mathbb Z}} % Gives you a shortcut for writing the blackboard Z for the integer numbers - \ZZ
\newcommand{\Rats}{\ensuremath{\mathbb Q}} % Gives you a shortcut for writing the blackboard Q for the rational numbers - \QQ
\newcommand{\Cplx}{\ensuremath{\mathbb C}} % Gives you a shortcut for writing the blackboard C for the complex numbers - \CC

% Make better absolute value bars and the norm symbol
\newcommand{\abs}[1]{\left|#1\right|}
\newcommand{\norm}[1]{\left|\left|\,#1\,\right|\right|}

%% Now make some equivalents that some people may prefer.
\let\RR\Reals
\let\NN\Nats
\let\II\Ints
\let\CC\Cplx
\let\ZZ\Ints

%Add a shortcut for \rightarrow
\let\ra\rightarrow

%Add a \diam command for diameter
\newcommand{\diam}{\text{diam}}

\usepackage{graphicx}
\usepackage[all,cmtip]{xy}
\def\net<#1>{\left<#1\right>}

\newcommand{\bbB}{\mathbb{B}}
\doclabel{Math F651: Homework 9}
\docdate{Due: April 7, 2023}
\begin{document}
\begin{problems}
\problem Suppose $\left<x\right>_{\alpha\in A}$ is a net
in $X$ that does not converge to $x\in X$.  Show that
there is an open set $U$ containing $x$ and a subnet
$\left<x_{\alpha_\beta}\right>_{\beta\in B}$ such that
$x_{\alpha_{\beta}}\not\in U$ for all $\beta\in B$.
Hint: For a particular `bad` $U$, 
take $B$ to be the entire subset of $A$ such that
$x_{\beta}\not\in U$ and show that $B$ is directed.
Then show that there is a natural increasing cofinal map
from $B$ to $A$.

\begin{proof}Suppose $\left<x_\alpha\right>_{\alpha\in A}$ is a net
    in $X$ that does not converge to $x\in X$. By definition there exists 
    a $U \in \mathcal{V}(x)$ such that $\left<x_\alpha\right>_{\alpha\in T(\alpha')} \not\subseteq U$ for 
    all $\alpha' \in A$. Let $B = \{\alpha \in A: x_\alpha \not\in U\}$. We will proceed by 
    showing that $B$ is directed with $\leq$ relation inherited from $A$. Clearly it's reflective, 
    and transitive, by $A$. Let $a, b \in B$, and note that $\left<x_\alpha\right>_{\alpha\in T(a)} \not\subseteq U$
    and $\left<x_\alpha\right>_{\alpha\in T(b)} \not\subseteq U$, since $x_a, x_b \not\in U$. Since $A$ is directed, there exist a $c \in A$ 
    such that $c \geq a, b$. Since $c \in A$ we know that $\langle x_c' \rangle_{c' \in T(c)} \not\subseteq U$, and therefore there 
    exists a $x_c^* \not\in U$. Note that by definition $c^* \in B$ and since $c^* \in T(c)$ it follows $c^* \geq c \geq a, b$ and thus $B$ is directed.   
    
    
    We will proceed to show that $f:B \to A$ defined by the identity is increasing and cofinal. Clearly this map is increasing, 
    since our directness was inherited from $A$. Let $a \in A$, and note that  $\left<x_\alpha\right>_{\alpha\in T(a)} \not\subseteq U$
    so there exists some $b \in T(a)$ such that $x_b \not\in U$. Note that $b \in B$ and $f(b) = b \geq a$. 

    Therefore $B$ defines a subnet $\langle x_{\alpha_\beta}\rangle$ which by construction has the property that $x_{\alpha_\beta} \not\in U$
    for all $\beta \in B$. 

\end{proof}




\problem Crossley 6.1
Show that the spaces $[0,1]$ and $(0,1)$ are homotopy equivalent
by finding an explicit homotopy equivalence and its inverse 
between the two spaces.

\begin{proof} 
    Let $f: [0, 1] \to (0, 1)$ be defined by $f(x) = \frac{1 + 2(x)}{4}$. Let $g: (0, 1) \to [0, 1]$ be the 
    identity map. 
    Consider $g \circ f$ which is defined by $g(f(x)) = \frac{1 + 2(x)}{4}$. We will show this function is homotopy equivalent 
    to the identity on $(0, 1)$ by exhibiting an explicit homotopy. Consider the function $H_1: I \times I \to I$ defined by, 
    \begin{equation*}
        H_1(x, t) = \frac{1 + 2(x)}{4}(1 - t) + x(t).
    \end{equation*}
    This function is continuous and note that $H_1(x, 0) =  \frac{1 + 2(x)}{4} = g(f(x))$ and $H_2(x, 1) = x$. 

    Consider $f \circ g$ which is defined by $f(g(x)) = \frac{1 + 2(x)}{4}$. We will show this function is homotopy equivalent 
    to the identity on $(0, 1)$ by exhibiting an explicit homotopy. Consider the function $H_2: (0, 1) \times I \to (0, 1)$ defined by,
    \begin{equation*}
        H_2(x, t) = \frac{1 + 2(x)}{4}(1 - t) + x(t).
    \end{equation*}
    This function is also continuous and note that $H_2(x, 0) =  \frac{1 + 2(x)}{4} = f(g(x))$ and $H_2(x, 1) = x$. 
\end{proof}


velcro



\newpage
\problem Crossley 6.4b

Suppose $f:X\rightarrow S^n$ is continuous and not surjective.
Show that it is homotopic to a constant map.
\begin{proof} Suppose $f:X\rightarrow S^n$ is continuous and not surjective. Let $p \not\in f(X)$ and 
    recall that $A = S^n\setminus \{p\}$ is homeomorphic to $\RR^n$ via the 
    stereographic projection map $\sigma$. Note that $\sigma(f(x))$ is a map from $X \to \RR^n$, and recall that 
    $\RR^n$ is contractible and therefore there exists a homotopy, namely $H(x, t) = \sigma(f(x))(1 - t) + q(t)$ with the property that 
    $H(x, 0) = \sigma(f(x))$ and $H(x, 1) = q$ where $q = \sigma(i)$ for some $i \in A$. Note that $\sigma^{-1}(H(x, t))$ is a function 
    defined from $X \times I \to A$ with the property that 
    \begin{equation*}
        \sigma^{-1}(H(x,t)) = \sigma^{-1}(\sigma(f(x))(1) + q(0)) = f(x)
    \end{equation*}
     and
     \begin{equation*}
        \sigma^{-1}(H(x,t)) = \sigma^{-1}(\sigma(f(x))(0) + q(1)) = \sigma^{-1}(q) = \sigma^{-1}(\sigma(i)) = i,
     \end{equation*}
    a constant map in $S^n$ since $i \in A \subseteq S^n$. 
    Since $\sigma^{-1}(H(x, t))$ is a composition of continuous functions it is also continuous and thus we have constructed 
    a homotopy from $f$ to a constant map. 
\end{proof}







\problem Crossley 6.5

Show by means of an explicit homotopy
that the map $f:S^1\rightarrow S^1$ given by $f(x,y)=(-x,-y)$
is homotopic to the identity.
\begin{proof} Suppose $f:S^1\rightarrow S^1$ given by $f(x,y)=(-x,-y)$. Note that $(x, y) = z$ for some $z \in \CC$ 
    and then the function is equivalent to $f(z) = e^{i\pi}z$. Clearly our desired homotopy is given by $H(z, t) = e^{i\pi(t)}z$
    since it is continuous and $H(z, 0) = e^0z = z$ and $H(z, 1) = e^{i\pi(1)}z = f(z)$.     
\end{proof}







%this is the set of homotopy classes from X to itself
\problem Show that a space $X$ is contractible if and only if 
$[X,X]$ consists of a single element.
\begin{proof}$(\Rightarrow)$ Suppose $X$ is contractible. By definition $X$ is homotopy equivalent to a one point space, 
    and therefore for $Y = \{p\}$ there exists function $c_p: X \to Y$ and $c_0: Y \to X$ such that $c_p \circ c_0$ is homotopic to $i_{Y}$
    and $c_0 \circ c_p$ is homotopic to $i_X$. Since $Y$ is a one point space $[Y:X]$ has only one homotopy class and therefore, 
\begin{align*}
    [f]  &= [f \circ i_X]\\
          &= [f] \circ [i_X]\\
         &= [f] \circ [c_0 \circ c_p]\\
         &= [f \circ c_0] \circ [c_p]\\
         &= [c_0] \circ [c_p]\\
         &= [g \circ c_0] \circ [c_p]\\
         &= [g] \circ [c_0 \circ c_p]\\
         &= [g] \circ [i_X]\\
         &= [g \circ i_X]\\
         &= [g]
\end{align*}
    
\end{proof}


\begin{proof}$(\Leftarrow)$ Suppose $[X,X]$ consists of a single element. Then the constant map for $p \in X$ is homotopic to the
    identity map in $X$. Consider the one point space $A = \{p\}$ and note that $c_p: X \to A$ and $i_X|_p$, the identity map
    restricted to $A$ when composed 
    as $c_p \circ i_X|_p$ give the identity in $A$ and $i_X|_p \circ c_p$ give a constant map in $X$, which is homotopic to the identity in $X$
    and therefore $X \sim A$. 
\end{proof}


\problem Suppose that $f, g: S^n \to S^n$ are continuous maps such that $f(x) \neq -g(x)$ for any $x \in S^n$. Prove that 
$f$ and $g$ are homotopic. 
\begin{proof} Consider the function $H:S^n \times I \to S^n$, 
    \begin{equation*}
        H(x, t) = \dfrac{f(x)(1 - t) + g(x)t}{|f(x)(1 - t) + g(x)t|}.
    \end{equation*}
    Note that since $f, g: S^n \to S^n$ we know that  $H(x, 0) = \frac{f(x)}{|f(x)|} = f(x)$ and $H(x, 1) = \frac{g(x)}{|g(x)|} = g(x)$.
    What is left to show is that $H(x, t)$ is continuous from $S^n \times I \to S^n$. 
    
    Note that $n(x) = \frac{x}{|x|}$ is a continuous function 
    from $R^n\setminus \{0\} \to S^n$. Let $c(x, t): S^1 \times I \to R^n$ be defined by $c(x, t) = f(x)(1 - t) + g(x)t$ and suppose to the contrary that $f(x)(1 - t) + g(x)t = 0$,
    that would imply that $g(x) = \frac{t - 1}{t}f(x)$, and since $f$ and $g$ are maps into $S^n$, taking the absolute value of 
    both sides gives, 
    \begin{align*}
        \left|g(x)\right| &= \left|\dfrac{t - 1}{t}\right|\left|f(x)\right|,\\
        1 &= \left|\dfrac{t - 1}{t}\right|,\\
        1 &= \dfrac{|t - 1|}{t},\\
        t &= \left|t - 1\right|.
    \end{align*}
    so $t = 1/2$. Substituting back into our equation we get $f(x) = -g(x)$ a contradiction. Thus $f(x)(1 - t) + g(x)t \neq 0$
    for all $(x, t) \in S^n \times I$. Clearly $c(x, t)$ is continuous, and since $c(x, t) \neq 0$ we know that $n(c(x, t)) = H(x, t)$ is continuous, as it is a composition of continuous functions. 
\end{proof}
                    

\end{problems}

\end{document}

