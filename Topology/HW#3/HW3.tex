\documentclass[minion]{homework651}
%% Feel free to add your own commonly used commands to this file.

\newcommand{\Reals}{\ensuremath{\mathbb R}}% Gives you a shortcut for writing the blackboard R for the real numbers - \RR
\newcommand{\Nats}{\ensuremath{\mathbb N}} % Gives you a shortcut for writing the blackboard N for the natural numbers - \NN
\newcommand{\Ints}{\ensuremath{\mathbb Z}} % Gives you a shortcut for writing the blackboard Z for the integer numbers - \ZZ
\newcommand{\Rats}{\ensuremath{\mathbb Q}} % Gives you a shortcut for writing the blackboard Q for the rational numbers - \QQ
\newcommand{\Cplx}{\ensuremath{\mathbb C}} % Gives you a shortcut for writing the blackboard C for the complex numbers - \CC

% Make better absolute value bars and the norm symbol
\newcommand{\abs}[1]{\left|#1\right|}
\newcommand{\norm}[1]{\left|\left|\,#1\,\right|\right|}

%% Now make some equivalents that some people may prefer.
\let\RR\Reals
\let\NN\Nats
\let\II\Ints
\let\CC\Cplx
\let\ZZ\Ints

%Add a shortcut for \rightarrow
\let\ra\rightarrow

%Add a \diam command for diameter
\newcommand{\diam}{\text{diam}}

\def\calB{\mathcal{B}}
\DeclareMathOperator{\Int}{\mathrm{Int}}


\doclabel{Math F651: Homework 3}
\docdate{Due: February 8, 2023}
\docauthor{Stefano Fochesatto}

\begin{document}


\begin{problems}

\problem  Suppose $\calB_1$ and $\calB_2$ are bases for topologies $\tau_1$
and $\tau_2$.  Show that $\tau_1\subseteq \tau_2$ if and only if
for every $B_1\in \calB_1$ and every $x\in B_1$ there is a $B_2\in \calB_2$
such that $x\in B_2\subseteq B_1$.

\begin{proof}($\Rightarrow$) Let $\calB_1$ and $\calB_2$ be bases for topologies $\tau_1$
    and $\tau_2$ and suppose $\tau_1\subseteq \tau_2$. Let $B_1 \in \calB_1$, and note that since 
    $\calB_1$ is a basis for $\tau_1$ we know that $\calB_1 \in \tau_1, \tau_2$. Since $\calB_1$ is 
    in $\tau_2$ and $\calB_2$ is a basis for $\tau_2$ it follows that for every $x\in B_1$ there is a $B_2\in \calB_2$
    such that $x\in B_2\subseteq B_1$. \\


    ($\Leftarrow$) Let $\calB_1$ and $\calB_2$ be bases for topologies $\tau_1$
    and $\tau_2$ and suppose that for every $B_1\in \calB_1$ and every $x\in B_1$ there is a $B_2\in \calB_2$
    such that $x\in B_2\subseteq B_1$. Let $U \in \tau_1$, and note that since $\calB_1$ is a basis for $\tau_1$ 
    we know that for some index set $I_1$ of $\calB_1$,
    \begin{equation*}
        U = \bigcup_{i \in I_1} B^i_1.
    \end{equation*}
    Note that by our supposition for every $B^i_1 \in I_1$ there exists an index set $I_i$ of  $\calB_2$ such that,
    \begin{equation*}
        B^i_1 = \bigcup_{j \in I_i} B^j_2.
    \end{equation*}
    Therefore we know that, 
    \begin{equation*}
        U = \bigcup_{i \in I_1} \bigcup_{j \in I_i} B^j_2.
    \end{equation*}
    Having expressed $U$ as a union of sets in $\calB_2$ we know that $u \in \tau_2$ and thus $\tau_1\subseteq \tau_2$.
\end{proof}


\problem Given a family $\{\tau_\alpha\}_{\alpha\in I}$ of topologies in $X$,
show that there is a unique smallest topology containing each $\tau_\alpha$.
Show also that there is a unique largest topology contained in each $\tau_\alpha$.
Take advantage of past work!
\begin{proof} Let $\{\tau_\alpha\}_{\alpha\in I}$ be a family of topologies in $X$.
    Let $B = \bigcup_{\alpha \in I} \tau_\alpha$. Now consider the pre-basis $\cal{B}$, a set consisting of all unions of finite intersections of elements of $B$. Let $\tau'$ be the topology generated by $\cal{B}$.
    Suppose that $\tau \subseteq \tau'$ such that $\tau$ contains each $\tau_\alpha$. Therefore $\tau$ must contain all elements of $B$, and since it's a topology it must be closed with respect to unions and finite intersections therefore it must also contain $\cal{B}$. Hence $\tau=\tau'$.

    

        Let $A = \bigcap_{\alpha \in I} \tau_\alpha$ and note that it is a topology contained in each $\tau_\alpha$. Showing $A$ is a topology, simply note that any union or finite intersection of elements in $A$ must have also been in every $\tau_\alpha$ since they are also topologies and therefore $A$ is closed with respect to unions and finite intersections. Suppose there exists some $\tau$ contained in each $\tau_\alpha$ such that $A \subseteq \tau$. Let $x \in \tau$ and note that $x \in \tau_\alpha$ for all $\alpha$, therefore by definition $x \in A$. 

\end{proof}


\problem Let $\calB=\{[a,b):a,b\in \Rats\}$.  Show that $\calB$ 
is a pre-basis
and hence generates a topology $\tau_\calB$.  Compare this topology to the 
lower-limit topology $\tau_\ell$.  In particular, determine if it is finer or coarser
or neither or both.
\begin{proof} Let $\calB=\{[a,b):a,b\in \Rats\}$. Clearly we can see that $\cup_{B \in \calB} B = \RR$.
    Let $[a,b), [c, d) \in \calB$ and consider some $x \in [a,b) \cap [c, d)$. If the intersection is non-empty
    either the clopen intervals overlap or one is contained in the other. In either case the resulted intersection is 
    another clopen interval $[y, z) \in \calB$, so finally $x \in [y, z) \subseteq [a,b) \cap [c, d)$  thus $\calB$ is a pre-basis. 

    Let $\tau_\calB$ be the topology generated by $\calB$. Note that the interval $[\pi, 1) \in \tau_\ell$ is not 
    open with respect to $\tau_\calB$ since there is no $[a,b) \in \calB$ such that $\pi \in [a,b) \subseteq [\pi, 1)$. 
    Thus $\tau_\calB$ is the coarser topology. 
    
\end{proof}

\problem \textbf{Problem 2-12} Suppose $X$ is a set, and $\mathcal{A} \subseteq \mathcal{P}(X)$ is any collection 
of subsets of $X$. Let $\tau \subseteq \mathcal{P}(X)$ be the collection of subsets consisting of $X$, $\emptyset$, 
and all unions of finite intersection of elements of $\mathcal{A}$. 
\begin{enumerate}
    \item[(a)] Show that $\tau$ is a topology. 
    \begin{proof} Suppose a set $X$, and $\mathcal{A} \subseteq \mathcal{P}(X)$ with $\tau \subseteq \mathcal{P}(X)$ be the collection of subsets consisting of $X$, $\emptyset$, 
        and all unions of finite intersection of elements of $\mathcal{A}$. By definition $X, \emptyset \in \tau$. 
        Let $\{U_i\}_I \subseteq \tau$ and note that for each $U_i$ there exists a collection $\{A_{i,j}\}_{J}$ where each $A_{i,j}$ is some finite intersection of the elements of $\mathcal{A}$ such that $U_i = \cup_{j \in J}A_{i, j}$. 

        By substitution we get, 
        \begin{equation*}
            \bigcup_{i \in I} U_i = \bigcup_{i \in I} \bigcup_{j \in J} A_{i,j} 
        \end{equation*}
        Note that this union is itself a union of finite intersections of elements of $\mathcal{A}$, thus $\tau$ is closed with respect to unions. 
        Let $\{U_i\}_I \subseteq \tau$ be a finite subset. Note that by substitution and associativity of intersection we get, 
        \begin{equation*}
            \bigcap_{i \in I} U_i = \bigcap_{i \in I}\bigcup_{j \in J} A_{i,j} =  \bigcup \bigcap_{\substack{i \in I\\ j \in J}} A_{i,j}.
        \end{equation*}
        Again we've managed to write our finite intersection as a union of finite intersections of elements of $\mathcal{A}$, thus $\tau$ is closed with respect to finite intersections. 
    \end{proof}




    \item[(b)] Show that $\tau$ is the coarsest topology for which all the sets in $\mathcal{A}$
    are open. 
    \begin{proof} Suppose there exists some topology $\tau'$ such that $\tau' \subseteq \tau$ and $\mathcal{A} \subseteq \tau'$. Let $U \in \tau$ and by definition we know that $U$ equal to the union of some finite intersection of the elements in $\mathcal{A}$. Well since $\tau'$ is a topology with $\mathcal{A} \subseteq \tau'$ we must have that $x \in \tau'$. Thus $\tau$ is the coarsest topology.
    \end{proof}


    \item[(c)] Let $Y$ be any topological space. Show that a map $f: Y \to X$ is continuous if and only if $f^{-1}(U)$
    is open in $Y$ for every $U \in \mathcal{A}$.
    \begin{proof} ($\Rightarrow$) Let $Y$ be any topological space and suppose the map $f: Y \to X$ is continuous. Note that by definition $U \in \mathcal{A}$ is open in $X$
        and by continuity we know that $f^{-1}(U)$ must be open in $Y$. 
    \end{proof}

    \begin{proof} ($\Leftarrow$) Let $Y$ be any topological space, consider the map 
        $f: Y \to X$ and suppose that for every $A \in \mathcal{A}$, $f^{-1}(A)$ is open in $Y$. Let $U \subseteq X$, and recall that by definition there exists a collection $\{\hat{A}_{j}\}_{J}$ where each $A_{j}$ is some finite intersection of the elements of $\mathcal{A}$ such that, 
        \begin{equation*}
        U = \bigcup_{j \in J}\hat{A}_{j} = \bigcup_{j \in J}\bigcap^n_{i \in I}A_{j,i}.
        \end{equation*}
        Considering the pre-image we find that, 
        \begin{equation*}
            f^{-1}(U) = f^{-1}\left(\bigcup_{j \in J}\bigcap^n_{i \in I}A_{j,i}\right) =\bigcup_{j \in J}\bigcap^n_{i \in I} f^{-1}\left(A_{j,i}\right)
        \end{equation*}
        Since $Y$ is a topological space and $f^{-1}(A)$ is open for every $A \in \mathcal{A}$ we can conclude that  $f^{-1}(U)$ is open in $Y$ and thus $f$ is continuous. 
    \end{proof}





    \item[(d)] Conclude that the topology generated by a pre-basis $\calB$ is the smallest topology in which every set from 
    $\calB$ is open.   
    \begin{proof} This conclusion comes directly from parts $(a)$ and $(b)$ of this problem. Note that $\tau$ from $(a)$ is the topology generated by a pre-basis. In $(b)$ we showed that $\tau$ is the coarsest (or smallest) topology for which all of the sub-basis elements are included.         
    \end{proof}
\end{enumerate} 


\problem \textbf{Problem 2-15} Let $X$ and $Y$ be topological spaces. 
\begin{enumerate}
    \item[(a)] Suppose $f:X \to Y$ is continuous and $p_n \to p$ in $X$. Show that $f(p_n) \to f(p)$ in Y.  (This was proved in last weeks homework.) 
    \begin{proof}
        Let $U \in \mathcal{V}(f(p))$ and note that since $f$ is continuous we know that 
        $f^{-1}(U)$ is open in $X$. Since $p_n \to p$ in $X$ and $f^{-1}(U) \in \mathcal{V}(p)$ there exists some $N \in \NN$ such that $p_n \in f^{-1}(U)$ for all 
        $n \geq N$. It then follows that $f(p_n) \in U$ for all $n \geq N$ and thus by definition $f(p_n) \to f(p)$.
    \end{proof}

    \item[(b)] Prove that if $X$ is first countable then the converse is true: if $f: X \to Y$ is a map such that $p_n \to p$ in $X$ implies $f(p_n) \to f(p)$ in $Y$, then $f$ is continuous. 
    \begin{proof}We will proceed by proving the contrapositive. 
        Suppose that $f:X \to Y$ is not continuous. Then there exists some $U \subseteq Y$ such that $f^{-1}(U)$ is not open in $X$. Therefore there exists some $x \in f^{-1}(U)$ such that for every $U' \in \mathcal{V}(x)$, $U' \not\subseteq f^{-1}(U)$. Choose one such $U'$ and since $X$ is first countable there exists a nested neighborhood basis $\{U'_k\}$ about $x$ such that for all $k$, $x \in U'_k \subseteq U'$. Choose $p_k \in U'_k$ such that $p_k \not\in f^{-1}(U)$. By construction we know that $p_k \to x$, yet clearly $f(p_k) \not\in f(x)$ since $f(x) \in U$ but $f(p_k) \not\in U$ for all $k$. 
        

        Thus we have constructed a sequence which converges in $X$, whose image does not converge in $Y$.
    \end{proof}
\end{enumerate}

\problem Let $A$ be a subset of a topological space $X$, and let $\calB$ be a basis for the topology. 
\begin{enumerate}
    \item[(a)] Show that $x \in \overline{A}$ if and only if for every $B \in \calB$ with $x \in B$, $B \cap A \neq \emptyset$. 
    \begin{proof} $(\Rightarrow)$ Suppose $x \in \overline{A}$. Observe that by definition $x$ is a contact point of $A$ and since all $B \in \calB$ are open, 
    $B \cap A \neq \emptyset$ is immediate. 
    \end{proof}

    \begin{proof}$(\Leftarrow)$ Suppose that for every $B \in \calB$ with $x \in B$, $B \cap A \neq \emptyset$. Let $U \subseteq X$ such that $x \in U$. Note that since $\calB$ is a basis there exists a collection of $\{B_i\}_{i \in I} \subseteq \calB$
    such that $U = \cup_{i \in I}B_i$. By our supposition we see that, 
    \begin{equation*}
        U \bigcap A = \left(\bigcup_{i \in I}B_i\right) \bigcap A = \bigcup_{i \in I} \left( B_i \bigcap A\right) \neq \emptyset
    \end{equation*} 
    \end{proof}

    \item[(b)] Show that $x\in\partial A$ if and only if for every $B\in\calB$  with $x\in B$, $B\cap A\neq\emptyset$
    and $B\cap A^c\neq \emptyset$

    \begin{proof}$(\Rightarrow)$ Suppose $x\in\partial A$. By our definition of $\partial A = \overline{A} \cap \overline{A^c}$ we know that $x \in \overline{A}, \overline{A^c}$. Therefore by $(a)$ we can conclude for every $B\in\calB$  with $x\in B$, $B\cap A\neq\emptyset$ and $B\cap A^c\neq \emptyset$.
    \end{proof}

    \begin{proof}$(\Leftarrow)$ Suppose for every $B\in\calB$ with $x\in B$, $B\cap A\neq\emptyset$ and $B\cap A^c\neq \emptyset$. Again by $(a)$ we can conclude that $x \in \overline{A}, \overline{A^c}$ and therefore $x \in \partial A$. 
    \end{proof}

    \item[(c)] Show that $\Int(A)\cap\partial A = \emptyset$ and $\overline{A}=\Int(A)\cup \partial A$. 
    \begin{proof} Suppose $x \in \partial A$. Let $U \in \mathcal{V}(x)$ and note that since  $x \in \partial A$ we 
        know that $U \cap A^c \neq \emptyset$ and since $\Int(A)$ is open we conclude that $x \not\in \Int(A)$. Thus 
        $\Int(A)\cap\partial A = \emptyset$.
    \end{proof}

    \begin{proof} Let $x \in \overline{A}$. By definition $x$ is a contact point of $A$ and therefore for all 
        $U \in \mathcal{V}(x)$ we know that $U \cap A \neq \emptyset$. If there exists a $U$ such that $U \subseteq A$ then 
        $x \in \Int(A)$ otherwise for all $U \in \mathcal{V}(x)$ we know that $U \cap A^c \neq \emptyset$ which implies $x \in \partial A$. Showing containment in the other direction comes from the fact that all points in $\Int(A) \cap \partial A$ are contact points of $A$.  
    \end{proof}

\end{enumerate}
 


\problem \textbf{Problem 2-20} Show that second countability, separability, and the Lindelöf property are all equivalent for metric space.\\

\begin{proof} ($2^{nd}$ countable $\Rightarrow$ Lindelöf) \emph{This proof was done in class}. 
    
    Suppose $X$ is a second countable metric space. Let $\{U_{\alpha}\}_{\alpha \in I}$ be an open cover for $X$. Since $X$ is 
    second countable we know that there exists $\{W_k\}$, a countable basis for $X$. Since $\{W_k\}$ is a basis we can consider the index set $K$, with the property that for all $k \in K$, $W_k \subseteq U_{\alpha}$ for some $\alpha \in I$. Therefore, for each $k \in K$ there exists some $\alpha_k$ such that $W_k \subseteq U_{\alpha_k}$. Clearly ${U_{\alpha_k}}_{k \in K}$ is a countable refinement of $\{U_{\alpha}\}_{\alpha \in I}$.
    Note that since $\{W_k\}$ is a basis, for every $x \in X$ we know that $x \in W_k \subseteq U_{\alpha_k}$ for some $k$ and therefore $\{U_{\alpha}\}_{\alpha \in I}$ is still an open cover of $x$.
\end{proof}


\begin{proof} ($2^{nd}$ countable $\Rightarrow$ separable) \emph{This proof was outlined in class}. 
    
    Suppose $X$ is a second countable metric space. Since $X$ is second countable we know that there exists $\{W_k\}$, a countable basis for $X$. Consider the set $\{p_k\}$ such that $p_k \in W_k$. Clearly $\{p_k\}$ is countable so we will proceed to show that $\{p_k\}$ is dense. Let $U \subseteq X$ be open and choose $x \in U$. Since $\{W_k\}$ is a basis we know that for some $k$, $x \in W_k \subseteq U$. Immediately it follows that $p_k \in W_k \subseteq U$ and thus $\{p_k\}$ is dense in $X$. 
\end{proof}


\begin{proof} (Lindelöf $\Rightarrow$ separable) \emph{Glen and I worked on this together with your help.} 
    
    Let $X$ be a Lindelöf metric space.
    For all $n \in \NN$ consider the sets $\mathcal{U}_n = \{B_{1/n}(x): x \in X\}$. Clearly each $\mathcal{U}_n$ is an open cover of $X$ and since $X$ is Lindelöf there exists countable subcovers $\mathcal{{U'}}_n$. Let $\{x^n_i\}_{i \in \NN}$ be the set of centers of balls in $\mathcal{{U'}}_n$. Note that since $\mathcal{{U'}}_n$ was a countable subcover, 
    $\{x^n_i\}_{i \in \NN}$ is also countable. Now consider the following set, 
    \begin{equation*}
        A = \bigcup_{n \in \NN}\bigcup_{i \in \NN} x^n_i.
    \end{equation*}
    Note that $A$ is a countable set. We will proceed by showing that $A$ is dense in $X$ by showing that for all $x \in X$, for every $\epsilon>0$, $B_\epsilon(x)\cap A \neq \emptyset$. Let $x \in X$ and $\epsilon > 0$, consider the subset of $A$ such that $n \geq 2/\epsilon$ and note that since points in this subset are at most $\epsilon/2$ distance apart there must exists some $x^n_i \in B_\epsilon(x)$. Hence $A$ is a countable dense subset of $X$.
\end{proof}


\begin{proof} (seperable $\Rightarrow$ $2^{nd}$ countable) 

    Suppose $X$ is a seperable metric space. Since $X$ is seperable there exists a set countable dense subset, $A$. 
    Now consider the countable set of open balls, 
    \begin{equation*}
        \mathcal{A} = \Biggl\{ B_{\frac{1}{n}}(x) \subseteq X : x \in A, n \in \NN \Biggr\}.
    \end{equation*}
    We will proceed by showing that this set is a basis for $X$. Let $\epsilon > 0$ and consider $B_{\epsilon}(x)$ for some $x \in X$. Since $A$ is dense there exists some $x' \in A$ such that $x' \in B_{\epsilon/4}(x)$. Note that by construction of $d(x, x') < \epsilon/4$ it follows that for all $B_{n}(x') \in \mathcal{A}$ with $n \geq 2/\epsilon$ we know $B_{n}(x') \subseteq B_{\epsilon}(x)$ since for all $x'' \in B_{n}(x')$, 
    \begin{equation*}
        d(x,x'') \leq d(x,x') + d(x', x'') = \epsilon/4 + \epsilon/2 < \epsilon.
    \end{equation*}
   Since the open balls about $x'$ have at most radius $\epsilon/2$ there must exists some $B_n(x')$ such that $x \in B_n(x') \subseteq B_\epsilon(x)$.
\end{proof}


\end{problems}

\end{document}
