\documentclass[minion]{homework651}
\include{hwextras}

\def\calB{\mathcal{B}}
\DeclareMathOperator{\Int}{\mathrm{Int}}

\doclabel{Math F651: Homework 5}
\docdate{Due: February 22, 2023}
%\docauthor{Your Name Here}

\begin{document}

Note: The book has Exercises, which are interspersed among the
prose, and Problems, which appear at the ends of the chapters.
It can be easy to confuse the two.  Exercises are denoted in blue.


\begin{problems}

\problem Suppose $\mathcal B_X$ and $\mathcal B_Y$ are bases for $X$ and $Y$ respectively.
Show that $\mathcal B = \{U\times V: U\in B_X, V\in B_Y\}$ is a basis for $X\times Y$.

\problem Suppose $A\subset X$ and $B\subset Y$.  Use the fact that
the characteristic property of the product topology is characteristic to
show that the subspace topology on $A\times B$ is the the same as its
topology as a product of subspaces.





\problem Show that $(X_1\times X_2)\times X_3$ is homeomorphic to $X_1\times X_2 \times X_3$.
You may not use the words ``open'' or ``closed'' at any point in your proof.  (\textit{Hint:} Use the Characteristic Property, Luke!)


\problem Prove the following.
\begin{subproblems}

\item A projection map from an arbitrary product space is an open map.

\item An arbitrary product of Hausdorff spaces is Hausdorff

\item A countable product of second countable spaces is second countable.

\end{subproblems}

\problem\textbf{Problem 3-8} Let $X$ denote the cartesian product of countably infinitely copies of $\RR$
endowed with the box topology. Define a map $f: \RR \to X$ by $f(x) = (x, x, x, \dots)$. Show that $f$ is 
not continuous even though each of its component functions is. 

\problem\textbf{Problem 3-9} Let $X$ be as in the preceding problem. Let $X^+\subseteq X$ be the subset consisting of sequences of 
strictly positive real numbers, and let $z$ denote the zero sequence, that is, the one whose terms are $x_i = 0$ for all $i$. 
Show that $z$ is in the closure of $X^+$, but there is no sequence of elements of $X^+$ converging to $z$. Then use the sequence lemma to conclude that $X$
is not first countable, and thus not metrizable.  

\problem\textbf{Problem 3-13 a} Suppose $X$ and $Y$ are topological spaces and $f: X \to Y$ is a continuous map. Prove that if $f$ admits 
a continuous left inverse, it is a topological embedding. 


\problem \exercise{3.61} Prove that a continuous surjective map $q: X \to Y$ is a quotient map if and only if it takes saturated open subsets to open subsets, 
or saturated closed subsets to closed subsets. 






\problem This is a heads up: problem 3-14 will be on the next homework.  Start thinking about it now.  
\end{problems}



\end{document}